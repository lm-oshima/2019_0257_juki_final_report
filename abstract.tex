\subsection{背景}
人・モノ・組織など、あらゆるものをつなげるICT・IoTの世界的な流れは、1つの業界に留まらずあらゆる産業において多大な影響を及ぼしつつある。それらは、既存プレーヤーの役割や主導権のシフト、異業種からの新規参入、新たな業態創出といった変革を促し、これまでにない産業・市場・投資・価値を産み出すだけでなく、既存の構造を破壊・再生さえも引き起こしている\footnote{総務省 平成30年版情報通信白書より}。
例えば製造業においては、旧来の垂直統合型・水平分業型から、データドリブン型のプラットフォーム構造やオープンイノベーションといった新しい枠組みが産まれるとともに、2017年度には我が国の在り方として「Connected Industries」のコンセプトが発信されるなど、変革が加速する素地が整いつつある\footnote{経済産業省 製造業をめぐる現状と政策課題 ~Connected Industriesの深化~}。

繊維産業に目を向けてみると、世界的には成長を維持しているものの、国内では衣類単価の減少・人材不足が、海外では人件費の高騰といった課題がある\footnote{経済産業省 繊維産業の課題と経済産業省の取組}。この課題に対し、高付加価値素材への注力に加え、JUKI株式会社によるスマートファイクトリー構想をはじめとしたICTの活用による新しい価値創造が進められており、「モノ」を作り売る産業から、モノから新たな「体験・価値」を創る産業へと移りつつある\footnote{「Value up2022」 2019-2021年度の中期経営計画}。

製造業における新たな「体験・価値」の1つとしては、製造工程から人を排除した完全なる自動化が考えられる。第一次産業革命以来、人の手によりモノの製造・検査・配送等が行われていたが、制御工学・工作機械の発展により、近年では製造工程の一部に人が関わらない自動化が図られている。この領域に、さらに人工知能・ロボティクス・IoTといった分野が加わり、人が知と経験を以って行う必要がある部分さえも機械に置き換わろうとしている。人を介さないことの価値は、人件費や人材確保などの側面に加え、品質を安定させること・時間/場所/環境を問わないこと、そして人がより生産性の高い業務に専念できることなどが挙げられる。この自動化を実現する手段の1つとして、人工知能分野では“Deep Learning”技術の研究開発が行われている。

\subsection{Deep Learningとは}
Deep Learningとは、人の脳が持つ神経ネットワークの一部分と類似した計算アルゴリズムであり、コンピュータ自身が処理のルールを学習できるものである。旧来の制御工学においては、人がルールを定義しコンピュータにプログラムしていた。このルールは非常に堅く・かつ人が無理なく設定できる数量であるため、実際の環境において発生しうる揺らぎにより定義されていない事象がプログラムに入力された場合、プログラムではその入力を適切に処理できず、予期せぬ動作・エラーが発生してしまう。一方、Deep Learningでは、バリエーションのある実データを使い、人がプログラムできない程の多数のパラメータ・ルールを“学習”させることで、人が未知の事象に対し経験を以って判別するように曖昧さを吸収するルールをコンピュータが作り出すことができる。

その違いとして、写真の中からペットボトルを見つけるプログラムを作る場合を考える。旧来のやり方では、図のようにペットボトルの特徴を定義し、それに合致する写真を選択する手法を取るであろう。例えば、大きさは280ml, 350ml, 500ml, 1L, 1.5L, 2Lとし、キャップは円形で形は四角柱や六角柱で、カバーの色は...と、かなりのパラメータを検討し定義づけていく。しかし、仮にそれらを入力できたとして、ペットボトルを斜めから撮影している・人が手に持っている・新発売の特殊な形状である...といった、定義から外れる条件下では、ペットボトルを認識できないであろう。仮にそれらをプログラムで定義しようとしても、データのバリエーションが無数に存在するため、実現することは非現実的である。一方で、Deep Learningでは、ペットボトルの持つ特徴量を学習を通じてコンピュータ自身がルール化しアルゴリズム内のパラメータとして記憶する。その結果、仮に学習していないペットボトルが写真に含まれていたとしても、その特徴量が学習内容とある程度一致するものであれば認識することが可能となる。この判断は、人の認識技術と近しい振る舞いであり、曖昧な・想定と異なる状況が発生したとしても適正に対処できうる可能性があるため、人の経験などで行なっていた業務をDeep Learningを使ったコンピュータに置き換えることができる。
\\
\begin{figure}[htbp]
  \begin{center}
    \includegraphics[width=140mm]{images/vsdl.png}
    \caption{旧来のプログラムとDeep Learningの違い}
    \label{fig:vsdl}
  \end{center}
\end{figure}


\subsection{目的}
本プロジェクトは、紡織機器・産業用ミシンにおいて、縫製自動化のためにDeep Learningが適用できるか予備検討を行うものである。ミシンにおいて縫製位置を自動で設定するには、理想的にはCADなどで設計された縫製図面を使用すれば良いと考えられるが、実際には、縫製途中において布地に歪みや位置ずれが発生するため、図面から算出された理想的な縫製位置とは異なる場所へ最適位置が移動してしまう。この位置ずれは、縫製生地の材質や生地の縫い場所などにより無数のパタンが発生し、旧来のプログラム手法のように、位置ずれのルールを定義するやり方では全てのケースを満足することができない。そこで、縫製時の写真を学習データに用いたDeep Learningモデルを作り、未知のデータ・位置ずれが発生した場合でも検出できるか検証を行う。

\subsection{ロードマップ}
本プロジェクトにおける開発ロードマップのイメージを図に示す。紡織機器を自動化させるためには、Deep Learningモデルの作成に加え、モデルを搭載するハードウェアの選定/開発/検証、カメラや紡織機器との接続検証、紡織工程や紡織機器製造工程を考慮した運用設計といった、システム全体の設計/開発/検証が必要となる。このうちDeep Learning部分では、モデルの精度・速度とハードウェア性能にトレードオフ関係があることに加え、外付けハードウェア機器を使うのか・既存システムのCPUを使うかなど、システム構成も大きく変わるため、システム設計とモデル開発を並行して実施する必要がある。今回の検討フェーズでは、本プロジェクトにおけるDeep Learning適用可否の評価を主として行うが、製品化・サービス化までの工程を検討し、着実に進めていく必要があると考えられる。
\\
\begin{figure}[htbp]
  \begin{center}
    \includegraphics[width=140mm]{images/roadmap.png}
    \caption{開発ロードマップ}
    \label{fig:roadmap}
  \end{center}
\end{figure}