%#!ptex2pdf -l
%#BIBTEX bibtex main
\documentclass[dvipdfmx,autodetect-engine,10pt,a4paper]{jsarticle}
\usepackage[deluxe]{otf}
\renewcommand{\headfont}{\gtfamily\sffamily\bfseries}
\usepackage[defaultsans]{droidsans}
\renewcommand*\familydefault{\sfdefault}
\usepackage[T1]{fontenc}
\renewcommand{\kanjifamilydefault}{\gtdefault}
\usepackage[dvipdfmx]{graphicx}
\usepackage{fancyhdr}
\usepackage{colortbl}
\usepackage{wallpaper}
\usepackage{bm}
\usepackage{amsmath}
\usepackage{url}
\usepackage{longtable}
\DeclareMathOperator*{\argmax}{arg\,max}
\DeclareMathOperator*{\argmin}{arg\,min}


\pagestyle{fancy}
\lhead{}
\chead{}
\rhead{}
\lfoot{LeapMind Inc. \copyright \, 2019.}
\cfoot{}
\rfoot{\thepage}
\renewcommand{\headrulewidth}{0.0pt}
\renewcommand{\footrulewidth}{0.4pt}


\begin{document}
\ThisLRCornerWallPaper{1.0}{images/footer.pdf}

\begin{center}
  \vspace*{20mm}
  \Huge 縫製位置検出のための\\
 \Huge ディープラーニングモデル検討 \par
  \vspace{10mm}
  \huge 2019年6月28日(金) \par
  \vspace{90mm}
  \includegraphics[width=8cm]{images/LM_Logo.pdf}
  \thispagestyle{empty}
  \clearpage
  \addtocounter{page}{-1}
  \end{center}
\newpage

\tableofcontents
\thispagestyle{empty}
\clearpage
\addtocounter{page}{-1}
\clearpage

\section{本プロジェクトの目的}
\subsection{背景}
人・モノ・組織など、あらゆるものをつなげるICT・IoTの世界的な流れは、1つの業界に留まらずあらゆる産業において多大な影響を及ぼしつつある。それらは、既存プレーヤーの役割や主導権のシフト、異業種からの新規参入、新たな業態創出といった変革を促し、これまでにない産業・市場・投資・価値を産み出すだけでなく、既存の構造を破壊・再生さえも引き起こしている\footnote{総務省 平成30年版情報通信白書より}。
例えば製造業においては、旧来の垂直統合型・水平分業型から、データドリブン型のプラットフォーム構造やオープンイノベーションといった新しい枠組みが産まれるとともに、2017年度には我が国の在り方として「Connected Industries」のコンセプトが発信されるなど、変革が加速する素地が整いつつある\footnote{経済産業省 製造業をめぐる現状と政策課題 ~Connected Industriesの深化~}。

繊維産業に目を向けてみると、世界的には成長を維持しているものの、国内では衣類単価の減少・人材不足が、海外では人件費の高騰といった課題がある\footnote{経済産業省 繊維産業の課題と経済産業省の取組}。この課題に対し、高付加価値素材への注力に加え、JUKI株式会社によるスマートファイクトリー構想をはじめとしたICTの活用による新しい価値創造が進められており、「モノ」を作り売る産業から、モノから新たな「体験・価値」を創る産業へと移りつつある\footnote{「Value up2022」 2019-2021年度の中期経営計画}。

製造業における新たな「体験・価値」の1つとしては、製造工程から人を排除した完全なる自動化が考えられる。第一次産業革命以来、人の手によりモノの製造・検査・配送等が行われていたが、制御工学・工作機械の発展により、近年では製造工程の一部に人が関わらない自動化が図られている。この領域に、さらに人工知能・ロボティクス・IoTといった分野が加わり、人が知と経験を以って行う必要がある部分さえも機械に置き換わろうとしている。人を介さないことの価値は、人件費や人材確保などの側面に加え、品質を安定させること・時間/場所/環境を問わないこと、そして人がより生産性の高い業務に専念できることなどが挙げられる。この自動化を実現する手段の1つとして、人工知能分野では“Deep Learning”技術の研究開発が行われている。

\subsection{Deep Learningとは}
Deep Learningとは、人の脳が持つ神経ネットワークの一部分と類似した計算アルゴリズムであり、コンピュータ自身が処理のルールを学習できるものである。旧来の制御工学においては、人がルールを定義しコンピュータにプログラムしていた。このルールは非常に堅く・かつ人が無理なく設定できる数量であるため、実際の環境において発生しうる揺らぎにより定義されていない事象がプログラムに入力された場合、プログラムではその入力を適切に処理できず、予期せぬ動作・エラーが発生してしまう。一方、Deep Learningでは、バリエーションのある実データを使い、人がプログラムできない程の多数のパラメータ・ルールを“学習”させることで、人が未知の事象に対し経験を以って判別するように曖昧さを吸収するルールをコンピュータが作り出すことができる。

その違いとして、写真の中からペットボトルを見つけるプログラムを作る場合を考える。旧来のやり方では、図のようにペットボトルの特徴を定義し、それに合致する写真を選択する手法を取るであろう。例えば、大きさは280ml, 350ml, 500ml, 1L, 1.5L, 2Lとし、キャップは円形で形は四角柱や六角柱で、カバーの色は...と、かなりのパラメータを検討し定義づけていく。しかし、仮にそれらを入力できたとして、ペットボトルを斜めから撮影している・人が手に持っている・新発売の特殊な形状である...といった、定義から外れる条件下では、ペットボトルを認識できないであろう。仮にそれらをプログラムで定義しようとしても、データのバリエーションが無数に存在するため、実現することは非現実的である。一方で、Deep Learningでは、ペットボトルの持つ特徴量を学習を通じてコンピュータ自身がルール化しアルゴリズム内のパラメータとして記憶する。その結果、仮に学習していないペットボトルが写真に含まれていたとしても、その特徴量が学習内容とある程度一致するものであれば認識することが可能となる。この判断は、人の認識技術と近しい振る舞いであり、曖昧な・想定と異なる状況が発生したとしても適正に対処できうる可能性があるため、人の経験などで行なっていた業務をDeep Learningを使ったコンピュータに置き換えることができる。
\\
\begin{figure}[htbp]
  \begin{center}
    \includegraphics[width=140mm]{images/vsdl.png}
    \caption{旧来のプログラムとDeep Learningの違い}
    \label{fig:vsdl}
  \end{center}
\end{figure}


\subsection{目的}
本プロジェクトは、紡織機器・産業用ミシンにおいて、縫製自動化のためにDeep Learningが適用できるか予備検討を行うものである。ミシンにおいて縫製位置を自動で設定するには、理想的にはCADなどで設計された縫製図面を使用すれば良いと考えられるが、実際には、縫製途中において布地に歪みや位置ずれが発生するため、図面から算出された理想的な縫製位置とは異なる場所へ最適位置が移動してしまう。この位置ずれは、縫製生地の材質や生地の縫い場所などにより無数のパタンが発生し、旧来のプログラム手法のように、位置ずれのルールを定義するやり方では全てのケースを満足することができない。そこで、縫製時の写真を学習データに用いたDeep Learningモデルを作り、未知のデータ・位置ずれが発生した場合でも検出できるか検証を行う。

\subsection{ロードマップ}
本プロジェクトにおける開発ロードマップのイメージを図に示す。紡織機器を自動化させるためには、Deep Learningモデルの作成に加え、モデルを搭載するハードウェアの選定/開発/検証、カメラや紡織機器との接続検証、紡織工程や紡織機器製造工程を考慮した運用設計といった、システム全体の設計/開発/検証が必要となる。このうちDeep Learning部分では、モデルの精度・速度とハードウェア性能にトレードオフ関係があることに加え、外付けハードウェア機器を使うのか・既存システムのCPUを使うかなど、システム構成も大きく変わるため、システム設計とモデル開発を並行して実施する必要がある。今回の検討フェーズでは、本プロジェクトにおけるDeep Learning適用可否の評価を主として行うが、製品化・サービス化までの工程を検討し、着実に進めていく必要があると考えられる。
\\
\begin{figure}[htbp]
  \begin{center}
    \includegraphics[width=140mm]{images/roadmap.png}
    \caption{開発ロードマップ}
    \label{fig:roadmap}
  \end{center}
\end{figure}

\clearpage
\section{Deep Learningによる縫製位置検出方法}
\subsection{画像からの縫製位置検出方法}
学習の対象となる画像例を図3に示す。この画像は、製造工程における縫製布の表面を拡大したものであり、点群のようなものは、布地表面の模様である。ここで、縫製位置は点群の中点を結んだ図4であり、このXを描くラインを自動で検出することが目的となる。例示した画像は比較的見やすいが、実際には後述するように、縫い跡や凹凸による影の発生などにより、様々な画像パタンが存在する。これらの異なるパタン全てに対応するためには、Deep Learningを使った位置検出を検討する意義が高いと言える。尚、実際にはX以外の形状・曲面を描くパタンもあるが、今回は学習の対象外とする。

\begin{figure}[htbp]
 \begin{minipage}{0.5\hsize}
  \begin{center}
   \includegraphics[width=60mm]{images/segmentation/sample01.png}
  \end{center}
  \caption{学習画像}
  \label{fig:one}
 \end{minipage}
 \begin{minipage}{0.5\hsize}
  \begin{center}
   \includegraphics[width=60mm]{images/segmentation/sample01-2.png}
  \end{center}
  \caption{縫製位置}
  \label{fig:two}
 \end{minipage}
\end{figure}


Deep Learningによる位置検出方法について検討を行う。縫製位置は、点群のエッジ部分からの中点に位置する。このため、例えば図5のように、点群の領域や、点群のエッジ部分をDeep Learningにより自動で検出することができれば、図6のように、その中間を画像処理により抽出することは容易に実現しうるものである。この点群領域を検出するため、今回はDeep Learningを用いたSemantic Segmentation手法を用いて検討する。

\begin{figure}[htbp]
 \begin{minipage}{0.5\hsize}
  \begin{center}
   \includegraphics[width=60mm]{images/segmentation/sample02.png}
  \end{center}
  \caption{点群領域の推定}
  \label{fig:three}
 \end{minipage}
 \begin{minipage}{0.5\hsize}
  \begin{center}
   \includegraphics[width=60mm]{images/segmentation/sample02-1.png}
  \end{center}
  \caption{縫製位置の検出}
  \label{fig:four}
 \end{minipage}
\end{figure}


\subsection{Segmentation手法}
Semantic Segmentationとは、画像の特徴を画素レベルで把握する手法のことである。その実現方法は多数あるが、Deep Learningを使った手法としてSegNet\footnote{http:\slash\slash mi.eng.cam.ac.uk\slash projects\slash segnet}について検討を行う。

SegNetとは、ケンブリッジ大学の研究グループにて提案された、画像をある画素単位で分割し、その特徴量を算出することで物体の種類に応じた領域の検出が可能となるオープンソースのDeep Learningアーキテクチャである。その構造を図7に示す\footnote{Vijay Badrinarayanan, Alex Kendall and Roberto Cipolla, "SegNet: A Deep Convolutional Encoder-Decorder Architecture for Image Segmentation," https:\slash\slash arxiv.org\slash pdf\slash 1511.00561.pdf}。SegNetは、EncoderとDecoderの処理ネットワークに分割できる。Encoderは、VGG16モデルをベースとした画像の局所特徴を抽出するための畳み込み層\footnote{Karen Simonyan and Andrew Zisserman, "Very Deep Convolutional Networks for Large-Scale Image Recognition," https:\slash\slash arxiv.org\slash pdf\slash 1409.1556.pdf}、ダウンサンプリングを行い画素単位を細かくしていくプーリング層、計算途上で勾配消失を防ぐBatch Normalization層などから構成されており、細分化した画素値から物体の持つ特徴量を学習していく。Decoderでは、Encoderによって得られた物体の種類と大まかな位置情報を持つ特徴量マップに対し、アップサンプリングと畳み込み処理を行うことで元の解像度へ対応づけを行っていく。ここで、図8に示すように、アップサンプリングではEncoderネットワークにおいて予め記憶したプーリング層の画素関係・Pooling Indicesを使い、より適した位置に特徴量を割り当てる。例えば、cを割り当てる際にプーリング層では図のような特徴量を持っていたとすると、割り当てられるべきは左下の領域となる。割り当てた特徴量は畳み込み演算により滑らかになり、最終段において特徴量を際立たせるSoftmax処理を行うことで、入力画像の特徴を画素単位で表示することができる。図中の例においては、道路の風景写真に対して、道路・路側帯・車といった物体の特徴をマップ化することで異なる色で塗り分けている。
\\
\begin{figure}[htbp]
  \begin{center}
    \includegraphics[width=140mm]{images/segmentation/segnet.png}
    \caption{SegNetアーキテクチャ}
    \label{fig:segnet}
  \end{center}
\end{figure}
\begin{figure}[htbp]
  \begin{center}
    \includegraphics[width=90mm]{images/segmentation/upsample.png}
    \caption{Upsample処理}
    \label{fig:upsample}
  \end{center}
\end{figure}

\newpage

\subsection{量子化ネットワーク}
LeapMindでは、Deep Learningモデルを軽量なエッジデバイスで動作させるため、モデルの量子化を行なっている。量子化とは、通常32bitや16bitの浮動小数点で表現されている重みや特徴量データを、より少ないbit数の固定小数や整数などで表現するものである。畳み込みを行うDeep Learningモデルでは、演算量の大部分を畳み込み処理、即ち乗算と加算が占めている。この演算過程において、処理すべき単位が高次の浮動小数点であった場合、必要な回路規模は固定小数点や整数と比べると格段に大きくなるため、消費電力やメモリ領域の増加、処理速度の低下といった問題を引き起こす。特に演算回路規模の少ないエッジデバイスにおいては、性能劣化だけでなく、そもそもハードウェアへの実装ができないケースも発生しうる。このため、精度劣化がさほど発生しない範囲内において、重みや特徴量データをより少ないbitで表現する量子化が有効な解決策となる。LeapMindでは、特にこの量子化に対して技術的な優位性があり、精度劣化を抑えつつ内部での処理を最小1bitで実施することができる。今回は、SegNetをベースとし量子化技術を加えたモデルを用いて学習・検証を実施する。

\clearpage
\section{データセット・学習手順}
\subsection{アノテーション}

Segmentationでは、教師あり学習を実施する。即ち、学習に使用する画像に対して、予め検出したい領域とそのカテゴリを用意する必要がある。画像に対し注釈をつけるという意味で、この作業をアノテーションと呼ぶ。アノテーションを行うためのツールとして、Microsoft 社の VoTT \footnote{VoTT:https:\slash\slash github.com\slash microsoft\slash VoTT}を使用した。

実際の学習画像に対するアノテーション例を図9に示す。縫製位置は画像中心部分のXを描く空白部分であるが、画像の両脇にも見切れた形でXが残っている。この部分がモデルの精度に影響する可能性があることを考慮して、今回はこの見切れた部分についてもアノテーションを実施した。また、領域の線は、Xに隣接する布地のドット模様の中心を通る形としている。
\\
\begin{figure}[h!]
\begin{center}
\includegraphics[width=7cm]{images/images_merge/ano_sample.png}
\caption{アノテーション例}
\end{center}
\end{figure}

\subsection{学習データセット}

Segmentationモデルを作成するために使用した学習データを記述する。縫製工程において撮影された布地の写真は、その特徴から”Normal” ”糸残り” ”縫い跡・影あり” ”縫い跡あり” の4つに分類される。それぞれの特徴は以下の通りである。

\begin{tabbing}
\hspace{25mm}\= \hspace{10mm} \kill
・Normal \>:生地表面の模様以外には写り込みが無い写真 \\
・糸残り \>	:中心部に縫製糸が残り見えている写真 \\
・縫い跡あり \>:中心部に縫製跡が残っている写真 \\
・縫い跡・影あり \>:中心部に縫製跡が残っており、かつ生地表面の凹凸による影が写り込んでいる写真\\
\end{tabbing}

\begin{figure}[h!]
\begin{center}
\includegraphics[width=160mm]{images/images_merge/four_features.png}
\caption{写真の特徴}
\end{center}
\end{figure}

実験では、それぞれの特徴を持った写真が一定数以上入るデータセットを用いてモデルの学習を行うとともに、学習データに含まれない画像を用いて推論を行い、モデルの評価を行った。この条件下で実験したものを“実験1”と定義する。また、実験1の結果として苦手とする特徴が発生した場合、その特徴を持つ画像を実験1で使用したデータセットに追加して再度実験を行うこととした。後者を“実験2”、“実験3”とし
\\
実験1では、受領した約1800枚の中からそれぞれの特徴が最低20枚以上含まれる形で161枚を抜粋し、データセットとして学習に使用した。実験2では、実験1の結果次第で苦手な特徴に対し10枚の追加を、実験3では20枚の追加を行うこととする。

\begin{tabbing}
\hspace{25mm}\= \hspace{10mm} \kill
・Normal \>:40枚\\
・糸残り \>	:26枚 \\
・縫い跡あり \>:74枚 \\
・縫い跡・影あり \>:22枚\\
\end{tabbing}

評価用のデータは、下記の内訳で37枚を使用した。同じ特徴内においても写真にバリエーションがあるため、ある程度のパタンの差異が出るようにデータを選択している。Normalでは比較的画像のバリエーションが少ないため、評価用データの枚数を他に比べ半分程度の分量としている。

\begin{tabbing}
\hspace{25mm}\= \hspace{10mm} \kill
・Normal \>:6枚\\
・糸残り \>	:11枚 \\
・縫い跡あり \>:10枚 \\
・縫い跡・影あり \>:10枚\\
\end{tabbing}

一般的にDeep Learningモデルを作成する場合、最も重要な要素の1つが、学習データの質と、質の高いデータの量である。今回のケースでは、学習に使う写真が実際の使用状況に即したものであり、複雑性があまりなく、かつある程度のバリエーションがあるため、質の面では一定の水準には達していると考えられる。学習・評価に使用したデータは、別途DVDに収めて提出する。

\clearpage
\section{モデル作成と評価指標}
\subsection{Segmentationモデル}
Segmentationモデルは、前述した通り、SegNetに量子化技術などを加えたLeapMind独自のものを使用している。これは“blueoil”に組み込まれているモデル“LmSegnetV1Quantize”と同等の構成である\footnote{blueoil:https:\slash\slash blue-oil.org}\footnote{blueoil:https:\slash\slash leapmind.io\slash news\slash content\slash 2460}。blueoilとは、LeapMindが持つ技術の粋を集めた組み込み機器向けDeep Learningのオープンソース・開発プラットフォームである。本プラットフォームで開発されたモデルは、FPGA, CPU, GPUなど、様々なデバイスへと実装が可能となっており、その機能は更に日々アップグレードされ続けている。

\subsection{学習・チューニング}
学習時における代表的なパラメーターとして以下を設定し用いた。

\begin{tabbing}
\hspace{40mm}\= \hspace{10mm} \kill
バッチサイズ \>: 32\\
エポック数 \>: 500\\
初期学習率 \>: 0.001\\
学習率変化 \>: 3-step-decay \\
イメージサイズ \>: 128 ×128\\
オーギュメンテーション \>: Blue, Brightness, Color, Contrast, FlipLeftRight, FlipTopBottom\\
\end{tabbing}

バッチサイズ とは、データセットをいくつかのサブセットに分けて1回学習する際のサブセット1つに含まれるデータ数であり、エポック数は、全てのデータセットが学習に使われる回数を示す。例えばデータセットが320枚の場合、上記の設定値では、1回の学習には32枚の画像が使われる。全てのデータセットが学習に1回使われるまでの学習回数は10回で、設定されたエポック数より総学習回数が5000回となる。学習率とは、学習時における誤差パラメータの変更量に相当し、学習工程の 1/3 毎に学習率を1/10 にする手法を取っている。イメージサイズは入力画像の画素数であり、受領した元画像のサイズ1280×960を小さく・正方形にリサイズしている。

オーギュメンテーション・データ拡張とは、元の学習データに対して画像処理を加え、新たな画像を作成することである。この処理自体では、データそのものの特徴量を増やすという効果はないものの、例えば検出したい物体が現実世界とは異なり左寄りになっている・昼間の画像が多い・障害物がなく綺麗に写りすぎている、といった条件において、データの偏りを緩和する効果などがある。今回は、製造工程での使用を念頭に入れ、主に画像の反転や輝度・色味の変化のみを実施している。

\subsection{評価指標}
Segmentationにおける学習結果の評価指標として、モデルにより予測された結果と、対応する教師データのアノテーション結果との比較を行い、検出すべき対象部位の重なり具合・IoU(Intersection over Union)を算出する。図のような例であれば、教師データと検出データで結果が一致している領域は5.0に対し、全ての領域が13であるため、約0.38となる。この指標においては、教師データと検出データが完全に一致していればIoUは1.0となるが、例え教師データの領域内に検出データが全て含まれている・または検出データの領域内に教師データが全て含まれているような状況においても、教師データ・検出データ双方が一致しない部分が少しでもあれば、結果は1.0より小さくなる。このため、IoUは高ければ高いほどモデルの精度は良いとは言えるものの、現実問題として完全に1.0となるケースはあまりなく、また、仮に結果が1.0より低い状況であったとしても実使用上問題が発生するかどうかは実際の推論結果を確認し分析することで判断していくこととなる。今回は、IoUと推論結果両者について評価を実施した。
\\
\begin{figure}[h!]
\begin{center}
\includegraphics[width=80mm]{images/images_merge/IoU.png}
\caption{IoU 計算方法}
\end{center}
\label{fig:iou}
\end{figure}


\clearpage
\section{評価結果}
\subsection{学習・推論結果}

実験1の学習結果を図12に示す。学習ステップに対するIoUの値は、400ステップ程で高い水準へと収束し、高い水準を維持したまま収束してることがわかる。この結果は学習が問題なく進んでいることを示しており、本取り組みにおけるDeep Learning の適応可能性は十分にあると考えられる。このモデルに対し、推論データを適用し評価を行った。推論結果の一覧は付録に掲載する。概ね所望の領域が十分検出できている一方、図13に示すように“糸残り”のデータに対しては、糸で囲まれた部分を検出領域として誤判定してしまうケースが見られている。このため、苦手な特徴を持つ"糸残り"のデータを学習に追加することで、検出結果の改善ができるか検討を行った。実験2では、実験1の学習データに“糸残り”の特徴を有する画像を10枚追加し、実験3においては、実験2に対し更に20枚追加し、モデルを作成し評価を行った。

\begin{figure}[htbp]
\begin{minipage}{0.5\hsize}
\begin{center}
\includegraphics[width=80mm]{images/images_merge/iou/iou1.png}
\end{center}
\caption{実験1:トレーニングステップに対するIoU}
\label{fig:iou1}
\end{minipage}
\begin{minipage}{0.5\hsize}
\begin{center}
\includegraphics[width=68mm]{images/images_merge/results/JUKI_train_validation_sep/20180907_152826496_739_113_img.png}
\end{center}
\caption{実験1:“糸残り”のデータに対する推論結果}
\label{fig:nokori1}
\end{minipage}
\end{figure} 

実験2,実験3のIoUならびに“糸残り”の特徴に対する推論結果を図に示す。実験1と実験2を比較すると、実験2におけるテストデータのIoUがより滑らかに収束していることが分かる。Deep Learningを使ったシステムを実用化するには、学習データに含まれない未知のデータに対する推論結果の性能、即ち汎化性能が重要となる。このため、実験2の方が実験1に比べ、より汎化性能が高い・より良いモデルとなったと言える。
一方、実験2と実験3を比較すると、実験3における学習データのIoUが揺らいでいる。これは、苦手なデータが増えたことによる影響と考えられる。しかし、テストデータの結果を比べてみると、実験2と実験3で差がなく、モデルとしてみるとより安定性が増している可能性が高い。実際、“糸残り”の特徴に対する推論結果を比べてみると、これまで発生していた誤り部分に改善が見られていることから、より汎化性能が高いモデルができたと考えられる。
また、画像のエッジ部分に凹凸が見られるが、この部分はSegmentationにおいて画像の縮小・拡大を行なっているためであり、スムージングなどの後処理により平滑化が可能となる。

尚、今回は少量の学習データにより実験を行っているが、実使用環境において汎化性能を向上させるには、質の高い学習データの量を増やすことに加え、学習時のパラメータである学習率やバッチサイズ などのチューニングを行うことが必要となる。

\begin{figure}[htbp]
\begin{minipage}{0.5\hsize}
\begin{center}
\includegraphics[width=80mm]{images/images_merge/iou/iou2.png}
\end{center}
\caption{実験2:トレーニングステップに対するIoU}
\label{fig:iou1}
\end{minipage}
\begin{minipage}{0.5\hsize}
\begin{center}
\includegraphics[width=68mm]{images/images_merge/results/JUKI_train_add20190625_validation_sep/20180907_152826496_739_113_img.png}
\end{center}
\caption{実験2:“糸残り”のデータに対する推論結果}
\label{fig:nokori1}
\end{minipage}
\end{figure} 

\begin{figure}[htbp]
\begin{minipage}{0.5\hsize}
\begin{center}
\includegraphics[width=80mm]{images/images_merge/iou/iou3.png}
\end{center}
\caption{実験3:トレーニングステップに対するIoU}
\label{fig:iou1}
\end{minipage}
\begin{minipage}{0.5\hsize}
\begin{center}
\includegraphics[width=68mm]{images/images_merge/results/JUKI_train_add20190626_validation_sep/20180907_152826496_739_113_img.png}
\end{center}
\caption{実験3:“糸残り”のデータに対する推論結果}
\label{fig:nokori1}
\end{minipage}
\end{figure} 

\newpage


\subsection{ハードウェアへの実装評価}

作成したモデルをハードウェアに実装し、処理速度について評価を行った。今回使用するハードウェアとしては、Terasic社により販売されているDE10-Nanoを使用した \footnote{DE10-Nano:https:\slash\slash DE10-Nano.terasic.com}。外観ならびに構成を図に示す。DE10-Nanoには、Intel製のローエンドFPGAであるCycloneV SE \footnote{Cyclone V SE:https:\slash\slash www.intel.co.jp\slash content\slash www\slash jp\slash ja\slash products\slash programmable\slash fpga\slash cyclone-v.html}が搭載されている。CycloneV SEには、同一パッケージ内に28nmプロセスで作成されたFPGAとデュアルコアのARM Cortex-A9が搭載されており、ARMプロセッサ側でLinux OSが動作する構成となっている。Deep Learningモデルを実装する際には、畳み込み演算などの演算処理をFPGA側で行い、画像のリサイズや特徴量の配置といった部分をARM側で行なっている。
処理速度の評価では、DE10-Nanoボード内に保存した画像に対し推論処理を100回実施し、推論結果が得られるまでの時間を計測した後、単位時間に推論処理ができる画像枚数・FPSの算出を行った。今回のモデルを実装し評価した結果、 2.76FPS (1秒間に2.76枚の画像推論処理が可能) が得られた。処理速度は、モデルの層構造・チャネルを調整することで改善が可能である。また、今回の推論画像は入力サイズを128×128 pixelとしているが、単純計算で画像サイズを96×96 pixelとすると処理速度は2倍となる。一方で、推論する画像の画素数が下がるため、精度劣化が生じる可能性がある。

実運用においては、カメラI/Fにおける撮像〜通信や、推論結果を縫製装置に戻すための電文作成・処理・通信といった部分が更に上乗せされるため、必要となる処理性能の算出と性能確保、ハードウェアの選定・コスト算出など、製品化・サービス化へ向けロードマップを踏まえた検討が必要になると考えられる。



\begin{figure}[htbp]
\begin{minipage}{0.5\hsize}
\begin{center}
\includegraphics[width=70mm]{images/de10.png}
\end{center}
\caption{DE10-Nanoボード・外観写真}
\label{fig:de10}
\end{minipage}
\begin{minipage}{0.5\hsize}
\begin{center}
\includegraphics[width=80mm]{images/de10_block.png}
\end{center}
\caption{DE10-Nanoボード・構成}
\label{fig:deblock}
\end{minipage}
\end{figure} 


\clearpage
\section{全体を通してのまとめ}

縫製の自動化へ向けて、縫製時の写真を学習データに用いたDeep Learningモデルを作り評価を行った。その結果、高いIoUと実データでの推論評価より、Deep Learningにより縫製位置を検出することができる可能性が高いという結論に至った。本モデルを搭載した自動縫製装置の実用化には、モデルの精度や処理速度向上のためのデータセット収集・チューニングに加え、CPU・軽量GPUといった実装するハードウェアの検討、製品・サービスの仕様検討、システム・運用設計といった部分が必要となる。本結果はその第一歩を踏み出すに十分な結果を得たと言える。

\clearpage
\section *{付録:推論結果一覧}
ここでは、左の列から順番に実験1, 2, 3全ての推論結果を列挙する。

\begin{center}
\noindent
\includegraphics[width=0.3\linewidth]{images/images_merge/results/JUKI_train_validation_sep/20180216_170703762_889_041_img.png}
\includegraphics[width=0.3\linewidth]{images/images_merge/results/JUKI_train_add20190625_validation_sep/20180216_170703762_889_041_img.png}
\includegraphics[width=0.3\linewidth]{images/images_merge/results/JUKI_train_add20190626_validation_sep/20180216_170703762_889_041_img.png}\\
\includegraphics[width=0.3\linewidth]{images/images_merge/results/JUKI_train_validation_sep/20180216_190227619_889_034_img.png}
\includegraphics[width=0.3\linewidth]{images/images_merge/results/JUKI_train_add20190625_validation_sep/20180216_190227619_889_034_img.png}
\includegraphics[width=0.3\linewidth]{images/images_merge/results/JUKI_train_add20190626_validation_sep/20180216_190227619_889_034_img.png}\\
\includegraphics[width=0.3\linewidth]{images/images_merge/results/JUKI_train_validation_sep/20180216_190352944_889_053_img.png}
\includegraphics[width=0.3\linewidth]{images/images_merge/results/JUKI_train_add20190625_validation_sep/20180216_190352944_889_053_img.png}
\includegraphics[width=0.3\linewidth]{images/images_merge/results/JUKI_train_add20190626_validation_sep/20180216_190352944_889_053_img.png}\\
\includegraphics[width=0.3\linewidth]{images/images_merge/results/JUKI_train_validation_sep/20180216_191447016_889_026_img.png}
\includegraphics[width=0.3\linewidth]{images/images_merge/results/JUKI_train_add20190625_validation_sep/20180216_191447016_889_026_img.png}
\includegraphics[width=0.3\linewidth]{images/images_merge/results/JUKI_train_add20190626_validation_sep/20180216_191447016_889_026_img.png}\\
\includegraphics[width=0.3\linewidth]{images/images_merge/results/JUKI_train_validation_sep/20180217_165020973_889_021_img.png}
\includegraphics[width=0.3\linewidth]{images/images_merge/results/JUKI_train_add20190625_validation_sep/20180217_165020973_889_021_img.png}
\includegraphics[width=0.3\linewidth]{images/images_merge/results/JUKI_train_add20190626_validation_sep/20180217_165020973_889_021_img.png}\\
\includegraphics[width=0.3\linewidth]{images/images_merge/results/JUKI_train_validation_sep/20180217_165030670_889_002_img.png}
\includegraphics[width=0.3\linewidth]{images/images_merge/results/JUKI_train_add20190625_validation_sep/20180217_165030670_889_002_img.png}
\includegraphics[width=0.3\linewidth]{images/images_merge/results/JUKI_train_add20190626_validation_sep/20180217_165030670_889_002_img.png}\\
\includegraphics[width=0.3\linewidth]{images/images_merge/results/JUKI_train_validation_sep/20180217_181224449_889_050_img.png}
\includegraphics[width=0.3\linewidth]{images/images_merge/results/JUKI_train_add20190625_validation_sep/20180217_181224449_889_050_img.png}
\includegraphics[width=0.3\linewidth]{images/images_merge/results/JUKI_train_add20190626_validation_sep/20180217_181224449_889_050_img.png}\\
\includegraphics[width=0.3\linewidth]{images/images_merge/results/JUKI_train_validation_sep/20180217_182356974_889_002_img.png}
\includegraphics[width=0.3\linewidth]{images/images_merge/results/JUKI_train_add20190625_validation_sep/20180217_182356974_889_002_img.png}
\includegraphics[width=0.3\linewidth]{images/images_merge/results/JUKI_train_add20190626_validation_sep/20180217_182356974_889_002_img.png}\\
\includegraphics[width=0.3\linewidth]{images/images_merge/results/JUKI_train_validation_sep/20180217_182602521_889_018_img.png}
\includegraphics[width=0.3\linewidth]{images/images_merge/results/JUKI_train_add20190625_validation_sep/20180217_182602521_889_018_img.png}
\includegraphics[width=0.3\linewidth]{images/images_merge/results/JUKI_train_add20190626_validation_sep/20180217_182602521_889_018_img.png}\\
\includegraphics[width=0.3\linewidth]{images/images_merge/results/JUKI_train_validation_sep/20180217_182718455_889_018_img.png}
\includegraphics[width=0.3\linewidth]{images/images_merge/results/JUKI_train_add20190625_validation_sep/20180217_182718455_889_018_img.png}
\includegraphics[width=0.3\linewidth]{images/images_merge/results/JUKI_train_add20190626_validation_sep/20180217_182718455_889_018_img.png}\\
\includegraphics[width=0.3\linewidth]{images/images_merge/results/JUKI_train_validation_sep/20180217_182827726_889_034_img.png}
\includegraphics[width=0.3\linewidth]{images/images_merge/results/JUKI_train_add20190625_validation_sep/20180217_182827726_889_034_img.png}
\includegraphics[width=0.3\linewidth]{images/images_merge/results/JUKI_train_add20190626_validation_sep/20180217_182827726_889_034_img.png}\\
\includegraphics[width=0.3\linewidth]{images/images_merge/results/JUKI_train_validation_sep/20180217_185624052_889_041_img.png}
\includegraphics[width=0.3\linewidth]{images/images_merge/results/JUKI_train_add20190625_validation_sep/20180217_185624052_889_041_img.png}
\includegraphics[width=0.3\linewidth]{images/images_merge/results/JUKI_train_add20190626_validation_sep/20180217_185624052_889_041_img.png}\\
\includegraphics[width=0.3\linewidth]{images/images_merge/results/JUKI_train_validation_sep/20180217_192926401_889_005_img.png}
\includegraphics[width=0.3\linewidth]{images/images_merge/results/JUKI_train_add20190625_validation_sep/20180217_192926401_889_005_img.png}
\includegraphics[width=0.3\linewidth]{images/images_merge/results/JUKI_train_add20190626_validation_sep/20180217_192926401_889_005_img.png}\\
\includegraphics[width=0.3\linewidth]{images/images_merge/results/JUKI_train_validation_sep/20180217_193117031_889_034_img.png}
\includegraphics[width=0.3\linewidth]{images/images_merge/results/JUKI_train_add20190625_validation_sep/20180217_193117031_889_034_img.png}
\includegraphics[width=0.3\linewidth]{images/images_merge/results/JUKI_train_add20190626_validation_sep/20180217_193117031_889_034_img.png}\\
\includegraphics[width=0.3\linewidth]{images/images_merge/results/JUKI_train_validation_sep/20180217_193322654_889_041_img.png}
\includegraphics[width=0.3\linewidth]{images/images_merge/results/JUKI_train_add20190625_validation_sep/20180217_193322654_889_041_img.png}
\includegraphics[width=0.3\linewidth]{images/images_merge/results/JUKI_train_add20190626_validation_sep/20180217_193322654_889_041_img.png}\\
\includegraphics[width=0.3\linewidth]{images/images_merge/results/JUKI_train_validation_sep/20180903_194213667_732_001_img.png}
\includegraphics[width=0.3\linewidth]{images/images_merge/results/JUKI_train_add20190625_validation_sep/20180903_194213667_732_001_img.png}
\includegraphics[width=0.3\linewidth]{images/images_merge/results/JUKI_train_add20190626_validation_sep/20180903_194213667_732_001_img.png}\\
\includegraphics[width=0.3\linewidth]{images/images_merge/results/JUKI_train_validation_sep/20180903_194230965_732_001_img.png}
\includegraphics[width=0.3\linewidth]{images/images_merge/results/JUKI_train_add20190625_validation_sep/20180903_194230965_732_001_img.png}
\includegraphics[width=0.3\linewidth]{images/images_merge/results/JUKI_train_add20190626_validation_sep/20180903_194230965_732_001_img.png}\\
\includegraphics[width=0.3\linewidth]{images/images_merge/results/JUKI_train_validation_sep/20180903_194259548_732_001_img.png}
\includegraphics[width=0.3\linewidth]{images/images_merge/results/JUKI_train_add20190625_validation_sep/20180903_194259548_732_001_img.png}
\includegraphics[width=0.3\linewidth]{images/images_merge/results/JUKI_train_add20190626_validation_sep/20180903_194259548_732_001_img.png}\\
\includegraphics[width=0.3\linewidth]{images/images_merge/results/JUKI_train_validation_sep/20180904_094556550_732_000_img.png}
\includegraphics[width=0.3\linewidth]{images/images_merge/results/JUKI_train_add20190625_validation_sep/20180904_094556550_732_000_img.png}
\includegraphics[width=0.3\linewidth]{images/images_merge/results/JUKI_train_add20190626_validation_sep/20180904_094556550_732_000_img.png}\\
\includegraphics[width=0.3\linewidth]{images/images_merge/results/JUKI_train_validation_sep/20180904_094558709_732_000_img.png}
\includegraphics[width=0.3\linewidth]{images/images_merge/results/JUKI_train_add20190625_validation_sep/20180904_094558709_732_000_img.png}
\includegraphics[width=0.3\linewidth]{images/images_merge/results/JUKI_train_add20190626_validation_sep/20180904_094558709_732_000_img.png}\\
\includegraphics[width=0.3\linewidth]{images/images_merge/results/JUKI_train_validation_sep/20180904_094635413_732_000_img.png}
\includegraphics[width=0.3\linewidth]{images/images_merge/results/JUKI_train_add20190625_validation_sep/20180904_094635413_732_000_img.png}
\includegraphics[width=0.3\linewidth]{images/images_merge/results/JUKI_train_add20190626_validation_sep/20180904_094635413_732_000_img.png}\\
\includegraphics[width=0.3\linewidth]{images/images_merge/results/JUKI_train_validation_sep/20180904_095249385_732_001_img.png}
\includegraphics[width=0.3\linewidth]{images/images_merge/results/JUKI_train_add20190625_validation_sep/20180904_095249385_732_001_img.png}
\includegraphics[width=0.3\linewidth]{images/images_merge/results/JUKI_train_add20190626_validation_sep/20180904_095249385_732_001_img.png}\\
\includegraphics[width=0.3\linewidth]{images/images_merge/results/JUKI_train_validation_sep/20180904_110207142_732_001_img.png}
\includegraphics[width=0.3\linewidth]{images/images_merge/results/JUKI_train_add20190625_validation_sep/20180904_110207142_732_001_img.png}
\includegraphics[width=0.3\linewidth]{images/images_merge/results/JUKI_train_add20190626_validation_sep/20180904_110207142_732_001_img.png}\\
\includegraphics[width=0.3\linewidth]{images/images_merge/results/JUKI_train_validation_sep/20180904_110816896_732_001_img.png}
\includegraphics[width=0.3\linewidth]{images/images_merge/results/JUKI_train_add20190625_validation_sep/20180904_110816896_732_001_img.png}
\includegraphics[width=0.3\linewidth]{images/images_merge/results/JUKI_train_add20190626_validation_sep/20180904_110816896_732_001_img.png}\\
\includegraphics[width=0.3\linewidth]{images/images_merge/results/JUKI_train_validation_sep/20180904_110946609_732_001_img.png}
\includegraphics[width=0.3\linewidth]{images/images_merge/results/JUKI_train_add20190625_validation_sep/20180904_110946609_732_001_img.png}
\includegraphics[width=0.3\linewidth]{images/images_merge/results/JUKI_train_add20190626_validation_sep/20180904_110946609_732_001_img.png}\\
\includegraphics[width=0.3\linewidth]{images/images_merge/results/JUKI_train_validation_sep/20180904_112917119_732_000_img.png}
\includegraphics[width=0.3\linewidth]{images/images_merge/results/JUKI_train_add20190625_validation_sep/20180904_112917119_732_000_img.png}
\includegraphics[width=0.3\linewidth]{images/images_merge/results/JUKI_train_add20190626_validation_sep/20180904_112917119_732_000_img.png}\\
\includegraphics[width=0.3\linewidth]{images/images_merge/results/JUKI_train_validation_sep/20180907_150409461_739_070_img.png}
\includegraphics[width=0.3\linewidth]{images/images_merge/results/JUKI_train_add20190625_validation_sep/20180907_150409461_739_070_img.png}
\includegraphics[width=0.3\linewidth]{images/images_merge/results/JUKI_train_add20190626_validation_sep/20180907_150409461_739_070_img.png}\\
\includegraphics[width=0.3\linewidth]{images/images_merge/results/JUKI_train_validation_sep/20180907_152633918_739_075_img.png}
\includegraphics[width=0.3\linewidth]{images/images_merge/results/JUKI_train_add20190625_validation_sep/20180907_152633918_739_075_img.png}
\includegraphics[width=0.3\linewidth]{images/images_merge/results/JUKI_train_add20190626_validation_sep/20180907_152633918_739_075_img.png}\\
\includegraphics[width=0.3\linewidth]{images/images_merge/results/JUKI_train_validation_sep/20180907_152744818_739_094_img.png}
\includegraphics[width=0.3\linewidth]{images/images_merge/results/JUKI_train_add20190625_validation_sep/20180907_152744818_739_094_img.png}
\includegraphics[width=0.3\linewidth]{images/images_merge/results/JUKI_train_add20190626_validation_sep/20180907_152744818_739_094_img.png}\\
\includegraphics[width=0.3\linewidth]{images/images_merge/results/JUKI_train_validation_sep/20180907_152826496_739_113_img.png}
\includegraphics[width=0.3\linewidth]{images/images_merge/results/JUKI_train_add20190625_validation_sep/20180907_152826496_739_113_img.png}
\includegraphics[width=0.3\linewidth]{images/images_merge/results/JUKI_train_add20190626_validation_sep/20180907_152826496_739_113_img.png}\\
\includegraphics[width=0.3\linewidth]{images/images_merge/results/JUKI_train_validation_sep/20180910_095532993_739_012_img.png}
\includegraphics[width=0.3\linewidth]{images/images_merge/results/JUKI_train_add20190625_validation_sep/20180910_095532993_739_012_img.png}
\includegraphics[width=0.3\linewidth]{images/images_merge/results/JUKI_train_add20190626_validation_sep/20180910_095532993_739_012_img.png}\\
\includegraphics[width=0.3\linewidth]{images/images_merge/results/JUKI_train_validation_sep/20180910_100046910_739_084_img.png}
\includegraphics[width=0.3\linewidth]{images/images_merge/results/JUKI_train_add20190625_validation_sep/20180910_100046910_739_084_img.png}
\includegraphics[width=0.3\linewidth]{images/images_merge/results/JUKI_train_add20190626_validation_sep/20180910_100046910_739_084_img.png}\\
\includegraphics[width=0.3\linewidth]{images/images_merge/results/JUKI_train_validation_sep/20180910_100143967_739_103_img.png}
\includegraphics[width=0.3\linewidth]{images/images_merge/results/JUKI_train_add20190625_validation_sep/20180910_100143967_739_103_img.png}
\includegraphics[width=0.3\linewidth]{images/images_merge/results/JUKI_train_add20190626_validation_sep/20180910_100143967_739_103_img.png}\\
\includegraphics[width=0.3\linewidth]{images/images_merge/results/JUKI_train_validation_sep/20180910_101352512_739_088_img.png}
\includegraphics[width=0.3\linewidth]{images/images_merge/results/JUKI_train_add20190625_validation_sep/20180910_101352512_739_088_img.png}
\includegraphics[width=0.3\linewidth]{images/images_merge/results/JUKI_train_add20190626_validation_sep/20180910_101352512_739_088_img.png}\\
\includegraphics[width=0.3\linewidth]{images/images_merge/results/JUKI_train_validation_sep/20180910_103328044_739_002_img.png}
\includegraphics[width=0.3\linewidth]{images/images_merge/results/JUKI_train_add20190625_validation_sep/20180910_103328044_739_002_img.png}
\includegraphics[width=0.3\linewidth]{images/images_merge/results/JUKI_train_add20190626_validation_sep/20180910_103328044_739_002_img.png}\\
\includegraphics[width=0.3\linewidth]{images/images_merge/results/JUKI_train_validation_sep/20180910_103637428_739_058_img.png}
\includegraphics[width=0.3\linewidth]{images/images_merge/results/JUKI_train_add20190625_validation_sep/20180910_103637428_739_058_img.png}
\includegraphics[width=0.3\linewidth]{images/images_merge/results/JUKI_train_add20190626_validation_sep/20180910_103637428_739_058_img.png}

\end{center}


\bibliography{citation}
\bibliographystyle{unsrt}

\end{document}